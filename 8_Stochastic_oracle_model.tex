%! TEX root = main.tex

\section{Stochastic oracle model}

Assume that we want to solve the following problem:
\begin{align}
	\begin{aligned}
		\text{minimize }\quad   & f(x) \\
		\text{subject to }\quad & x\in \mathcal{X},
	\end{aligned}
	\label{def:stochastic problem}
\end{align}
where $f$ is a convex function, possibly not known. The first order \emph{stochastic} oracle model defines the framework where given $x\in\mathcal{X}$ an oracle yields back a random variable $G$ that is an unbiased estimator of a subgradient of $f$ at $x$, namely, $\E[G] \in \partial f(x)$. If $X$ is a random variable, the oracle yields back a random variable $G$ that is an unbiased estimator of a subgradient of $f$ at $X$ conditionally on $X$, namely, $\E[G|X] \in \partial f(X)$.

\subsubsection*{$L$-Lipschitz functions}
We analyze the behavior of the projected gradient descent algorithm to solve problem \eqref{def:stochastic problem} when we replace exact knowledge of subgradients of $f$ with unbiased estimates of them. For a given initial point $X_1\in\X$, possibly random, and a given collection of step sizes $\eta_1,\eta_2,\ldots$, the stochastic projected gradient descent is defined by the sequence of random variables generated according to the following update:
\begin{align*}
Y_{s+1} &= X_s - \eta_s G_s,\\
X_{s+1} &= \Pi_{\X}(Y_{s+1}).
\end{align*}
Here, conditionally on $X_s$, the random variable $G_s$ is an unbiased estimator of the subgradient of the function $f$ at $X_s$, namely,
$$
	\E[G_s|X_s] \in\partial f(X_s).
$$
We state and prove the convergence of this method in the $L$-Lipschitz case.

\begin{theorem}[$L$-Lipschitz Stochastic Projected Subgradient Descent]
\label{thm:projectedgradLstoch}
Let $f$ be convex. Suppose $\X$ is contained in a Euclidean ball of radius $B$ centred at $X_1$. Suppose that for every $X \in \X$ the stochastic oracle yields an unbiased estimator of the subgradient of $f$ at $X$ bounded in $L_2$, namely, $\E[G |X]\in \partial f(X)$ and $\E[\|G\|^2] \leq L^2$. Then, the projected subgradient descent method with $\eta_s\equiv\eta = \frac{B}{L\sqrt{t}}$ satisfies 
\begin{equation} \label{eqn:proj_subgrad_t stoch}
\E f\left(\frac{1}{t} \sum_{s=1}^t X_s\right) - f(x^*) \leq \frac{LB}{\sqrt{t}}.
\end{equation}
\end{theorem}

\begin{proof}
For any $1 \leq s \leq t$ we have
\begin{align*}
	f(X_s) - f(x^*)
	\le \E [G_s|X_s]^T(X_s - x^*)
	= \E [G_s^T(X_s - x^*)|X_s].
\end{align*}
Proceeding as in the proof of Theorem \ref{thm:projectedgradL}, we find
\begin{align*}
	G_s^T(X_s - x^*)
	\le \frac{1}{2\eta} \left( \|X_s - x^*\|^2 - \|X_{s+1} - x^*\|^2 \right) + \frac{\eta}{2} \|G_s\|^2.
\end{align*}
Taking the expectation, we get
\begin{align*}
	\E f(X_s) - f(x^*) &\leq 
	\E [G_s^T(X_s - x^*)]
	\le \frac{1}{2\eta} \left( \E\|X_s - x^*\|^2 - \E\|X_{s+1} - x^*\|^2 \right) + \frac{\eta}{2} \E[\|G_s\|^2],
\end{align*}
and using the assumption $\E[\|G_s\|^2]\le L^2$ we get
\[
\frac{1}{t} \sum_{s=1}^t (\E f(X_s) - f(x^*)) \leq \frac{1}{2 \eta t} \left( \E\| X_1 - x^* \|^2 - \E\| X_{t+1} - x^* \|^2 \right) + \frac{\eta}{2} L^2 \leq \frac{B^2}{2 \eta t} + \frac{\eta L^2 }{2}.
\]
Selecting $\eta = \frac{B}{L\sqrt{t}}$ to minimize the right-hand side of the above inequality gives the first result in \eqref{eqn:proj_subgrad_t} (since $f\left(\frac{1}{t}\sum_{s=1}^t X_s \right) \leq \frac{1}{t} \sum_{s=1}^t f(X_s)$ by Jensen's inequality).
\end{proof}

Theorem \ref{thm:projectedgradLstoch} shows that, in expectation, the stochastic projected subgradient descent method yields the same convergence guarantees as the deterministic counterpart analyzed in Theorem \ref{thm:projectedgradL}.  In particular, the oracle complexity is the same\footnote{Note that different notions of accuracy are used, however, as we consider the expected value of the stochastic method.}: to get an accuracy $\varepsilon$, both methods requires $O(1/\varepsilon^2)$ calls to their respective oracles. The main advantage of the stochastic version lies in the fact that in some applications the \emph{computational} complexity involved in having access to a stochastic oracle is much cheaper than in the determinist case. We now show that the stochastic model yields substantial computational saving in machine learning.

\subsection{Multiple passes over the data}
TO DO.

\subsection{Single pass over the data}
We now show how the first order stochastic oracle model allows us to deal directly with expected risk minimization without invoking the empirical risk minimization paradigm.
Let us recall the original problem that motivates us in Section \ref{sec:intro}, namely, the expected risk minimization:
\begin{align*}
	\begin{aligned}
		\text{minimize }\quad   & r(x) = \E\ell(x^T\Phi(W),Y) \\
		\text{subject to }\quad & x\in \mathcal{X}.
	\end{aligned}
\end{align*}
The assumption here is that we know the loss function $\ell$ (indeed, we can choose it!) and the constraint set $\mathcal{X}$, but we do not know the distribution of $(W,Y)$. We only have access to $m$ i.i.d.\ samples $(W_1,Y_1),\ldots,(W_m,Y_m)$ from this unknown distribution. So we do not know the function $r$ that we want to minimize, and in particular we can not operate in the deterministic first order oracle model discussed in the previous weeks as for a given $x\in\mathcal{X}$ we do not have access to a subgradient in $\partial r(x)$. On the other hand, given $x\in\mathcal{X}$ we can have access to an unbiased estimator of a subgradient of $r$ evaluated at $x$. In fact, given the function $x \rightarrow R_i(x) := \ell(x^T\Phi(W_i),Y_i)$, which is known to us, we can compute a subgradient of $R_i$ at $x$. This is a random variable $G_i\in \partial R_i(x)$ that satisfies $\E G_i \in \partial r(x)$. The same is true if we want to get an estimate of the subgradient of $r$ at $X\in\X$ when $X$ is a random variable \emph{independent} of $(W_i,Y_i)$. In fact, in this case we can evaluate a subgradient of $R_i$ at $X$, and this random variable $G_i\in \partial R_i(X)$ satisfies $\E[G_i|X]\in r(X)$.\footnote{Note that if $X$ is random, then there are two sources of randomness in $G_i$: one source is $X$ itself, the other is the data point $(W_i,Y_i)$. The statement $\E[G_i|X]\in r(X)$ holds if $X$ and $(W_i,Y_i)$ are independent.} Hence, we satisfy the assumption of the first order stochastic oracle model. At the same time, as we have $m$ independent data points at out disposal, we can have access to at most $m$ independent unbiased estimators of subgradients evaluated at possibly different locations in $\mathcal{X}$. In other words, the requirement of independence restricts us to a \emph{single pass} over the data, which is not as general as in the stochastic block model defined above where we can have how many queries to the oracle as we want.

It is easy to check that under the assumptions of Proposition \ref{prop:Lip-stats}, Theorem \ref{thm:projectedgradLstoch} yields the following convergence guarantees for the stochastic projected descent method:
$$
	\E r(\hat X_m) - r(x^*) \leq \frac{GL_\textrm{loss}B}{\sqrt{m}},
$$
with $\hat X_m = \frac{1}{m}\sum_{i=1}^m X_i$. In other word, we can apply the stochastic gradient descent algorithm to \emph{directly} minimize the expected risk $r$, without invoking the empirical risk minimization paradigm, i.e., without breaking the problem into statistics and optimization as outlined in Section \ref{sec:intro}. The computational savings are clear. If computing a subgradient for each functions $R_i$ costs $O(1)$, then the computational complexity of stochastic gradient descent to achieve precision $\varepsilon$ (in expectation) is of order $O(1/\varepsilon^2)$. On the other hand, if we apply the deterministic gradient descent method to minimize the empirical risk $R$ up to precision $1/\sqrt{m}$, as discussed in Section \ref{sec:intro} and in Remark \ref{rem:optimization}, then we see that the computational complexity is $O(m/\varepsilon^2)$, as we need $O(1/\varepsilon^2)$ calls to the deterministic oracle but each call costs $O(m)$ base iterations, as $R(x) := \frac{1}{m} \sum_{i=1}^m R_i(x)$.